\begin{savequote}[75mm] 
Nulla facilisi. In vel sem. Morbi id urna in diam dignissim feugiat. Proin molestie tortor eu velit. Aliquam erat volutpat. Nullam ultrices, diam tempus vulputate egestas, eros pede varius leo.
\qauthor{Quoteauthor Lastname} 
\end{savequote}

\chapter{Introduction}
Research is about increasing the stock of knowledge, solving problems people face, devising ways to improve aspects of human lives, etc. As a researcher, it is sometimes helpful to be aware of distinctions
between different geographical regions. There are numerous reasons why distinctions could exist, and some distinctions are consequences of other basic distinctions. As
an example, the availability of
infrastructure or lack thereof results in differences in the way technology is used in different areas. We have observed online media sharing become
popular with the arrival of services such as Dropbox, Google+, Google Drive, etc. However, Walton et al\cite{RefWorks:26} have shown that in places such as
Khayelitsha (Cape Town, South Africa), co-located phone use surpassed online sharing. They observed that the media stored on the mobile devices became public personae, that is, it played the role of social media 'profiles'\cite[p. 403]{RefWorks:26}.\newline

The aforementioned services are great examples of cloud computing in use. The cloud (or more formally, cloud computing) can be viewed as the treatment of computing as a utility. Cachin et al\cite{cachin2009trusting} define cloud computing as being the flexible online data storage services, ranging from passive ones, such as online archiving, to active ones, such as collaboration and social networking \cite[p. 81]{cachin2009trusting}
Cloud computing platforms play an essential role to modern lives and the economy. The Google clould platform provides services to companies like Coca Cola, Khan Academy, Ubisoft and others. As far as computer science is concerned, the cloud is a useful computing paradigm from the perspective of a developer
as it removes the costs associated with buying hardware. It also removes other complications associated with hosting one's own over-the-Internet services. It is not as attractive, however, from the perspective of a mobile device user as it has a number of faults. Essentially, cloud computing is well suited for devices connected to networks with fast Internet speeds. It still faces challenges in mobile devices in a number of areas due to weak
network signals and bandwidth-induced delays. These challenges face mobile devices due to the distance of the cloud from the mobile device user. They are worth solving as they may deter some individuals from using cloud based applications.\newline

The technologies that power cloud computing can be used on a smaller scale through embedded systems. This results in the transformation of the cloud from a global panopticon to a hyper-localized and ad-hoc instantiation of the cloud. These instantiations are referred to as cloudlets. They address the issues of weak network signals and bandwidth-induced delays due to their proximity to the mobile device users. It is important to note that cloudlets have been explored through other technologies such as the HP-XW8200 workstation. Ling et al\cite{ling2011ar} used multiple HP-XW8200 workstations with two 3.2GHz Intel Xeonprocessors and 2GB RAM to evaluate the practicality of augmented reality cloudlets for mobile computing. They did find that their architecture was faster than the traditional approach of using the cloud.\newline

This report, on the other hand, details the investigation of a Raspberry Pi for hosting a cloudlet. The use of a Raspberry Pi introduces the idea of affordable and portable cloudlets. Portability is important because among its benefits; it allows friends to have a channel which can support co-located interactions wherever they are. It need not be connected to the Internet therefore it can even support co-located interactions in areas where there is little to no mobile network coverage. The use of mobile device when co-located may seem counter-intuitive, however, Harper et al\cite{RefWorks:15} (as cited by Reitmaier and Benz et al\cite[p. 381]{reitmaier2013designing}) has reported that people want to use mobile devices when co-located as means to enrich social interactions. It is then essential to understand and support co-located interactions.\newline

The use of the Raspberry Pi provides new possibilities but it also introduces memory and computation limitations as compared to the cloud. The Raspberry Pi is a single board computer developed by the Raspberry Pi Foundation. It was designed for the purpose of being an affordable computer that will make it possible to teach computer science in high schools. There are three models which are available. These are model A, B and B+. They differ in specifications and those specifications can be found at \url{ http://www.raspberrypi.org/products/}.

\section {Problem statement and aims}

There are numerous aspects the current cloud computing paradigm is lacking in, these limitations carry over to general cloudlets. The following are the limitations or aspects which the project will attempt to address:

  \begin{itemize}
  \itemsep0em
  \item \textbf{Information Ownership, Control and Physical security}: Data centers for cloud computing service providers can be located in any country in the world. Users
  of cloud computing services may be concerned about where their information is stored due to privacy laws and government organizations in different countries. The proposed
  system should afford users the ability to control where the physical location of their data and who has access to it.

   \item \textbf{Latency}: Cloud computing is not suited for most areas in third world countries due to poor mobile network connectivity. In certain cases, cell phone providers
   have high charges. These two factors therefore limit the use of cloud services for mobile devices users. The cloudlet should reduce the latency associated with communication
   between the mobile device and the cloudlet.

  \item \textbf{Resource constraints}: The cloud has abundant storage available. This has the implication that cloud providers can make use of a variety of database management
  systems and algorithms. They can also
  be able store large quantities of data. The proposed cloudlet, however, will operate on a single board computer. This has the implication that the chosen algorithms and
  technologies need to use resources effectively.
  Also, the files which are to be stored on the cloudlet will need to compressed to ensure optimal memory usage.

 \end{itemize}
 
 In essence, the project aims to (1) develop lightweight communication model
between clients and the cloudlet, (2) investigate suitable databases for a cloudlet instantiated using an embedded system, (3) investigate the practicality of using an inexpensive WiFi adapter for creating a cloudlet network, (4) investigate a combination of tools that effectively use computer resources and (5) create a robust cloudlet with a file service capable of handling frequent connections and disconnections.


\section{Research Questions}

\begin{itemize}
%\item Can we create an effective common sandbox for data?\\

%Application data in Android and other operating systems is shared between the operating system and the application. This creates a challenge when creating applications for sharing ephemeral data. The need for such services can be seen
%through the success of Snapchat, the application which allows users to send temporary multimedia to a controlled list of recipients. In the context of our cloudlet architecture, these applications should store data on the cloudlet for the consumption of all people who have access to it. It should not be possible for other users to store copies of shared content due because the operating system enabled them to have access to the files. We investigate the effectiveness of using the Raspberry Pi as a common sandbox.

\item Can an single board computer be practical for creating cloudlets, taking in consideration battery consumption and network range?\\

The primary users of cloudlets are mobile device users. Mobile device users are likely to go in and out of range. This means that the cloudlet
should be able to quickly detect connections and disconnections. Also, the portability of the raspberry pi means that it should cover a relatively large range to cater for the mobility of its users. We aim to investigate if the battery will be drained quickly when certain tasks are performed and if the range of the cloudlet is practical. It is vital that we determine how long would a Raspberry Pi using a portable power source would last, what is the maximum distance covered by an inexpensive WiFi adapter and how is bandwidth affected.
\end{itemize}

\section{Work allocation}
The expected outcomes of the project are a cloudlet platform and an Android application that will interface with the file sharing service that will run on the cloudlet. The Android application explores the possibility of using the Raspberry Pi as a data sandbox. It is also used to determine which conceptual metaphors are appropriate for presenting ephemeral data to users. The platform capable of hosting numerous services and file sharing service are to be developed by Zola Mahlaza. The Android application which is a client for the file sharing service will be developed by Jarvis Mutakha.

\section{Outline}
Chapter \ref{chaptertwo} will discuss the existing literature on the challenges and success of cloudlets, technologies which are well suited for a cloudlet and ways to address the resource constraints faced by cloudlets hosted on embedded systems. Chapter \ref{chapterthree} presents the uses of cloudlets which will determine how we evaluate the cloudlet as being practical. Chapter \ref{chapterfour} presents the design choices made and Chapter \ref{chapterfive} presents the implementation of the cloudlet platform and the how a cloudlet service can be created. Chapter \ref{chaptersix} and chapter \ref{chapterseven} reveal the test results and main results of the projects respectively.